 \section{Descripción del Problema}
	
\begin{frame}{Descripción del Problema}
	\begin{block}{Manipulación de información del usuario}
	\begin{itemize}
	  \item El desarrollador Android no tiene cómo definir políticas de seguridad para
	regular el flujo de información de sus aplicaciones.
	\item Complejidad para prevenir fugas de información del usuario.
	\end{itemize}
	\end{block}
	\pause
	\begin{block}{ Reporte McAffe}
		\begin{itemize}
	  	\item Aplicaciones Android invasivas de la privacidad del usuario.
	  	\item No toda aplicación invasiva contiene malware.
	  	\item De las aplicaciones que más vulneran la privacidad del usuario 35 \%
	  	contienen malware.
		\end{itemize}
	\end{block}
\end{frame}

\begin{frame}{Descripción del Problema}
\begin{block}{Limitaciones de la API}
\begin{itemize}
  \item Políticas de control de acceso de la API.
  \item Regular el acceso a recursos protegidos.
  \item No hacen seguimiento al flujo de información.
\end{itemize}
\end{block}
\pause
\begin{block}{Propuestas existentes}
Data-Flow con técnicás de análisis tainting\newline
\begin{itemize}
	  \item  Se hace seguimiento a los datos marcados.
	  \item Ejemplo: FlowDroid
\end{itemize}	
\end{block}
\end{frame}

\begin{frame}{Descripción del Problema}
\begin{block}{Propuestas existentes}
Flujo de información con técnicas Program Dependence Graphs(PDG)\newline
En qué consisten, de forma general.\newline
\begin{itemize}
	  \item  Ejemplo: Joana.
	  \item 
\end{itemize}	
\end{block}
\end{frame}


\begin{frame}{Descripción del Problema}
\begin{block}{Propuestas existentes limitaciones}
	
\end{block}
\begin{block}{Propuestas existentes limitaciones}

\end{block}
\end{frame}
