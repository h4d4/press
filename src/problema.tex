 \section{Descripción del Problema}

\begin{frame}{Descripción del Problema}
	\begin{block}{Manipulación de información del usuario}
	\begin{itemize}
	\item El desarrollador Android no tiene cómo definir políticas de seguridad para
	regular el flujo de información de sus aplicaciones.\pause
	\item Complejidad para prevenir fugas de información del usuario.
	\end{itemize}
	\end{block}
	\pause
	\begin{block}{Reporte McAffe}
	\begin{itemize}
	\item Aplicaciones Android invasivas de la privacidad del usuario.\pause
	\item No toda aplicación invasiva contiene malware.\pause
	\item De las aplicaciones que más vulneran la privacidad del usuario 35 \%
	contienen malware.
	\end{itemize}
	\end{block}
\end{frame}

\begin{frame}{Descripción del Problema}
\begin{block}{Limitaciones de la API}
\begin{itemize}
  \item Políticas de control de acceso de la API.\pause
  \item Regular el acceso a recursos protegidos.\pause
  \item No hacen seguimiento al flujo de información.
\end{itemize}
\end{block}
\pause
\begin{block}{Propuestas existentes}
\begin{itemize}
  \item Análisis estático y análisis dinámico.\pause
  \item Análisis dinámico: actuales caminos de ejecución.\pause
  \item Análisis estático: es posible incluir todos los caminos de ejecución.
\end{itemize}
\end{block}
\end{frame}

% \begin{frame}{Descripción del Problema}
% \begin{block}{Propuestas existentes}
% Data-Flow con técnicas de análisis tainting.
% \begin{itemize}
% 	  \item Se hace seguimiento a los datos marcados.
% 	  \item No incluye todos los posibles caminos de ejecución.
% 	  \item Ejemplo: FlowDroid
% \end{itemize}
% \end{block}
% \end{frame}
% % 
% \begin{frame}{Descripción del Problema}
% \begin{block}{Propuestas existentes}
% Flujo de información con técnicas Program Dependence Graphs(PDG).
% \begin{itemize}
%  	  \item Los PDG proveen una representación del programa que se analiza.
% 	  \item Análisis de flujos de información del programa de principio a fin.
% 	  \item Incluye todos los posibles caminos de ejecución.
% 	  \item Ejemplo: Joana.
% \end{itemize}
% \end{block}
% \end{frame}
\begin{frame}{Descripción del Problema}
\begin{block}{Propuestas existentes: FlowDroid}
\begin{itemize}
	  \item Data-Flow con técnicas de análisis tainting.
	  \item No incluye todos los posibles caminos de ejecución.
	  \item No permite definir políticas de seguridad.
\end{itemize}
\end{block}
\pause
\begin{block}{Propuestas existentes: Joana}
\begin{itemize}
  	  \item Flujo de información con técnicas Program Dependence Graphs(PDG). 
	  \item Incluye todos los posibles caminos de ejecución.
\end{itemize}
\end{block}
\end{frame}

\begin{frame}{Descripción del Problema}
\begin{block}{Enfoque Propuestas existentes}
\begin{itemize}
  \item Precisión y eficiencia del análisis.
  \item Identificar fugas de información. 
  \item Aplicativos de terceros ya implementados.
\end{itemize}
\end{block}
\end{frame}
% 
% 
% \begin{frame}{Descripción del Problema}
% \begin{block}{Propuestas existentes limitaciones}
% 	
% \end{block}
% \begin{block}{Propuestas existentes limitaciones}
% 
% \end{block}
% \end{frame}
