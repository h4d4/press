\section{Background}
\subsection{}
\begin{frame}{Background}
	\begin{block}{Técnicas de análisis}
	\begin{itemize}
	  \pause
	  \item Análisis estático. \pause 
	  \item Análisis dinámico. \pause
	\end{itemize}
	\end{block}
	\pause
	\begin{block}{Técnicas utilizadas en análisis estático}
	\begin{itemize}
	  \pause
	  \item Técnicas de flujo de datos. \pause 
	  \item Técnicas de flujo de control. \pause
	  \item Security Typed languages.
	\end{itemize}
	\end{block}
\end{frame}
\begin{frame}{Background}
	\begin{block}{Aplicaciones Android}
	\begin{itemize}
	  \pause
	  \item Aplicación Java con interfaces descritas en XML.\pause
	  \item Framework Android.\pause
	  \item Componentes de
	  aplicación: Activity, Service, Broadcast, Content Providers.\pause
	\end{itemize}
	\pause
	\end{block}
	\begin{block}{Sistema de anotaciones en Jif}
	\begin{itemize}
	  \pause
	  \item Lenguaje tipado de seguridad.\pause
	  \item Extensiones de seguridad para el lenguaje Java.\pause
	  \item Restricciones para uso de la información.\pause
	  \item Label checking.
	\end{itemize}
	\end{block}
\end{frame}
\begin{frame}{Background}
	\begin{block}{DML de JIF}
		Elementos del modelo de anotación:
		\begin{itemize}
		  \item Principals
		  \item Políticas
		  \item Labels
		\end{itemize}
	\end{block}
\end{frame}
\begin{frame}[fragile]{Background}
% 	\pause
	\begin{block}{Principals}
		Autoridad sobre un sistema o programa(Alice, Bob, Chunck)
	\end{block}
	\pause
	\begin{block}{Políticas}
		\{owner: reader list\}\ u \{owner: writer list\}
	\end{block}
	\pause
	\begin{block}{Labels} 
		Políticas de seguridad que se adicionan a las expresiones del programa.
		\begin{lstlisting}[style=base]
			int@{Alice:}@ code;
		\end{lstlisting}
	\end{block}
\end{frame}